\documentclass[a4paper,titlepage,11pt,twosides,floatssmall]{mwrep}
\usepackage[left=2.5cm,right=2.5cm,top=2.5cm,bottom=2.5cm]{geometry}
\usepackage[OT1]{fontenc}
\usepackage{polski}
\usepackage{amsmath}
\usepackage{amsfonts}
\usepackage{amssymb}
\usepackage{graphicx}
\usepackage{float}
\usepackage{url}
\usepackage{tikz}
\usetikzlibrary{arrows,calc,decorations.markings,math,arrows.meta}
\usepackage{rotating}
\usepackage[percent]{overpic}
\usepackage[cp1250]{inputenc}
\usepackage{xcolor}
\usepackage{colortbl}
\usepackage{pgfplots}
\usetikzlibrary{pgfplots.groupplots}
\usepackage{listings}
\usepackage{matlab-prettifier}
\usepackage{enumitem,amssymb}
\definecolor{szary}{rgb}{0.95,0.95,0.95}
\definecolor{niebieski}{rgb}{0,0.447,0.741}
\definecolor{czerwony}{rgb}{0.85,0.325,0.098}

\usepackage{siunitx}
%\usepackage{gensymb}
\sisetup{detect-weight,exponent-product=\cdot,output-decimal-marker={,},per-mode=symbol,binary-units=true,range-phrase={-},range-units=single}
\SendSettingsToPgf
%konfiguracje pakietu listings
\lstset{
	backgroundcolor=\color{szary},
	frame=single,
	breaklines=true,
}
\lstdefinestyle{customlatex}{
	basicstyle=\footnotesize\ttfamily,
	%basicstyle=\small\ttfamily,
}
\lstdefinestyle{customc}{
	breaklines=true,
	frame=tb,
	language=C,
	xleftmargin=0pt,
	showstringspaces=false,
	basicstyle=\small\ttfamily,
	keywordstyle=\bfseries\color{green!40!black},
	commentstyle=\itshape\color{purple!40!black},
	identifierstyle=\color{blue},
	stringstyle=\color{orange},
}
\lstdefinestyle{custommatlab}{
	captionpos=t,
	breaklines=true,
	frame=tb,
	xleftmargin=0pt,
	language=matlab,
	showstringspaces=false,
	basicstyle=\footnotesize\ttfamily,
	%basicstyle=\scriptsize\ttfamily,
	keywordstyle=\bfseries\color{green!40!black},
	commentstyle=\itshape\color{purple!40!black},
	identifierstyle=\color{blue},
	stringstyle=\color{orange},
}

%wymiar tekstu (bez �ywej paginy)
\textwidth 160mm \textheight 247mm

%ustawienia pakietu pgfplots
\pgfplotsset{
tick label style={font=\scriptsize},
label style={font=\small},
legend style={font=\small},
title style={font=\small}
}

\def\figurename{Rys.}
\def\tablename{Tab.}

%konfiguracja liczby p�ywaj�cych element�w
\setcounter{topnumber}{0}%2
\setcounter{bottomnumber}{3}%1
\setcounter{totalnumber}{5}%3
\renewcommand{\textfraction}{0.01}%0.2
\renewcommand{\topfraction}{0.95}%0.7
\renewcommand{\bottomfraction}{0.95}%0.3
\renewcommand{\floatpagefraction}{0.35}%0.5
\renewcommand\thesection{\arabic{section}}

\begin{document}
\frenchspacing
\pagestyle{uheadings}

%strona tytu�owa
\title{\bf Sprawozdanie z projektu i �wiczenia laboratoryjnego nr 1, zadanie nr 1\vskip 0.1cm}
\author{Hubert Kozubek, Przemys�aw Michalczewski}
\date{2021}

\makeatletter
\renewcommand{\maketitle}{\begin{titlepage}
\begin{center}{\LARGE {\bf
Wydzia� Elektroniki i Technik Informacyjnych}}\\
\vspace{0.4cm}
{\LARGE {\bf Politechnika Warszawska}}\\
\vspace{0.3cm}
\end{center}
\vspace{5cm}
\begin{center}
{\bf \LARGE Projektowanie uk�ad�w w sterowania\\ (projekt grupowy) \vskip 0.1cm}
\end{center}
\vspace{1cm}
\begin{center}
{\bf \LARGE \@title}
\end{center}
\vspace{2cm}
\begin{center}
{\bf \Large \@author \par}
\end{center}
\vspace*{\stretch{6}}
\begin{center}
\bf{\large{Warszawa, \@date\vskip 0.1cm}}
\end{center}
\end{titlepage}
}
\makeatother

\maketitle

\tableofcontents

\section{Cele projektu i laboratori�w}
Celem niniejszego laboratorium oraz projektu by�o zaprojektowanie oraz implementacja
algorytm�w regulacji jednowymiarowego procesu na grzewczym stanowisku laboratoryjnym.
%Stanowsko grzewcze sk�ada si� grza�ki G1, wentylatora W1 oraz czujnika temperatury T1

\section{Przebieg laboratorium}
Przez ca�y czas trakcie trwania laboratorium moc wentylatora W1 powinna by�
ustawiona 50\%.

\subsection{Zad 1}
W pierwszej kolejno�ci nale�a�o odczyta� warto�� temperatury odczytanej z 
termometru T1 w wyznaczonym punkcie pracy G1 = 26. Po ustawieniu mocy grza�ki
i odczekaniu, a� temperatura T1 ustabilizuje si� (rysunek), warto�� termometru T1 wynosi�a
\num{31.12} �C. 

\subsection{Zad 2}
W tej cz�ci laboratorium nale�a�o przeprowadzi� eksperyment dla 3 r�nych warto�ci mocy grza�ki G1.
Warto�ci G1 dla kt�rych zosta�y przeprowadzone eksperymenty to G1 = 36, G1 = 46 oraz G1 = 56. (rysunki)

\end{document}
